% Define document class
\documentclass[twocolumn]{aastex631}
\usepackage{showyourwork}

\newcommand{\dd}{\mathrm{d}}
\newcommand{\diff}[2]{\frac{\dd #1}{\dd #2}}
\newcommand{\MSun}{M_\odot}

% Begin!
\begin{document}

% Title
\title{LISA can measure the rate of eccentric binary black holes (that cannot be) observed by ground-based detectors}

% Author list
\author{@katiebreivik}

% Abstract with filler text
\begin{abstract}
    hype words about lisa, hype words about eccentricity, hype words about bbhs
\end{abstract}

% Main body with filler text
\section{Introduction}
\label{sec:intro}

The environmnetal origins of BBHs are highly degenerate with the eccentricity of the binaries themselves. Thus, a measurement of the rate of eccentric BBHs provides the ability to constrain the relative contributions of isolated and dynamical formation envrionments. 

\section{Rate Derivation}
\label{sec:rates}
Under the assumptions that most BBHs form in orbits with orbital frequencies below $\sim 1~\rm{mHz}$ and that the emission of GWs is the only driver of angular momentum loss in BBHs between the mHz and kHz GW regime, the rate of evolution of BBHs from the mHz to the kHz band can be described analytically using the \citet{Peters1964} evolution equations. Averaging the energy from GW emission, we can write the orbital separation and eccentricity evolution as a function of time as

\begin{equation}
    \Big\langle \frac{\mathrm{d}a}{\mathrm{d}t} \Big\rangle = - \frac{64}{5} \frac{G^3 m_1 m_2 (m_1 + m_2)}{c^5 a^3 (1 - e^2)^{7/2}} \Big(1 + \frac{73}{24} e^2 + \frac{37}{96} e^4 \Big)
    \label{eq:dadt}
\end{equation}

\noindent and 

\begin{equation}
    \Big\langle\frac{\mathrm{d}e}{\mathrm{d}t}\Big\rangle = -\frac{304}{15} e \frac{G^3 m_1 m_2 (m_1 + m_2)}{c^5 a^4 (1-e^2)^{7/2}} \Big( 1 + \frac{121}{304} e^{2}\Big).
    \label{eq:dedt}
\end{equation}

Using Kepler's third law, the orbital separation time evolution can be converted to an orbital frequency evolution 

\begin{equation}
    \Big\langle\frac{\mathrm{d}f}{\mathrm{d}t}\Big\rangle = \frac{48 n}{5 \pi} \frac{\Big(G \mathcal{M}_c \Big)^{5/3}}{c^5} (2 \pi f_{\rm orb})^{11/3} F(e), 
    \label{eq:fdot}
\end{equation}

\noindent where 

\begin{equation}
    F(e) = \frac{1 + \frac{73}{24} e^2 + \frac{37}{96} e^4}{(1 - e^2)^{7/2}}.
    \label{eq:eccentricity_enhancement_factor}
\end{equation}

The comoving merger rate density per unit chirp mass, $\mathcal{M}_c$ of BBHs can then be written as

\begin{equation}
    \frac{\mathrm{d}^2n}{\mathrm{d}\mathcal{M}_c \mathrm{d}t} = \mathcal{R}(\mathcal{M}_c) p(\mathcal{M}_c),
    \label{eq:merger_rate_density}
\end{equation}

\noindent where $p(\mathcal{M}_c)$ is the chirp mass distribution observed by LVK. 

For cirular binaries, this can be converted into a number of BBHs per unit mass, redshift, and frequency as

\begin{equation}
    \frac{\mathrm{d}^3N}{\mathrm{d}\mathcal{M}_c \mathrm{d}z \mathrm{d}f} = \frac{\mathrm{d}^2n}{\mathrm{d}\mathcal{M}_c \mathrm{d}t} \frac{\mathrm{d}V}{\mathrm{d}z} \frac{\mathrm{d}t}{\mathrm{d}f},
    \label{eq:rate_per_freq}
\end{equation}

where $\mathrm{d}t / \mathrm{d}f$ is just the inverse of equation \ref{eq:fdot}.

\section{Relating Merger Rates to Binary Rates at Various Eccentricities}

Let 
\begin{equation}
    \diff{N}{M_c \dd e \dd V \dd t} 
\end{equation}
be the compact object differential merger rate per chirp mass, per eccentricity,
per comoving volume, and per time.  This is the quantity that LIGO measures or
constrains, at least for compact objects in the range $1 \, \MSun \lesssim M_c
\lesssim 100 \, \MSun$.  The merger rate is a function of the compact binary
properties and also spacetime location (though homogeneity and isotropy
assumptions restrict this to a dependence on time), but we suppress these
arguments for simplicity except where necessary.  We can convert this to a
binary rate per orbital frequency via the following argument.  Let 
\begin{equation}
    T_\mathrm{merge}\left( M_c, e, f \right)
\end{equation}
be the time until a binary with chirp mass $M_c$, eccentricity $e$,
and orbital frequency $f$ mergers.  Then the compact binary rate per unit frequency at any given time is
obtained as an integral of the merger rate over the \emph{future} mergers of the
binaries occupying an infinitesimal range of frequencies $\dd f$:
\begin{multline}
    \diff{N}{M_c \dd e \dd V \dd f}\left( t \right) = \int_{t}^{\infty} \dd t' \, \diff{N}{M_c \dd e \dd V \dd t'} \\ \times \delta\left( \left( t' - t \right) - T_\mathrm{merger}\left( M_c, e, f \right) \right) \frac{\partial T_\mathrm{merger}}{\partial f}.
\end{multline}
The Dirac delta function ensures that the only merger times that contribute to
the integral are those for a binary of chirp mass $M_c$, eccentricity $e$, and
orbital frequency $f$ at time $t$.  The final Jacobian factor converts from a
density in \emph{time} (implied by the delta function) to a density in
\emph{frequency}.  The integral is trivial; assuming that $T_\mathrm{merger}$ is
much smaller than any timescale on which the merger rate evolves, the binary
rate per frequency at time $t$ is just the merger rate (in the very near future)
times the Jacobian factor to translate a density in time to a density in
frequency:
\begin{equation}
    \label{eq:mergers-per-frequency}
    \diff{N}{M_c \dd e \dd V \dd f} = \diff{N}{M_c \dd e \dd V \dd t} \frac{\partial T_\mathrm{merger}}{\partial f}.
\end{equation}

Note that $e$ in Eq.\ \eqref{eq:mergers-per-frequency}, appearing in the
denominator of the rate densities and (implicitly) in their arguments and the
arguments of $T_\mathrm{merger}$ refers to the eccentricity of the system
\emph{at merger}.  It is natural to write the rate density per eccentricity
\emph{at the time when the orbital frequency is $f$}.  There is a natural
mapping from merger eccentricity to eccentricity at frequency $f$, defined
(implicitly) by Eqs.\ \eqref{eq:dedt}, and \eqref{eq:fdot}; let
$e'$ be the eccentricity at frequency $f$ and 
\begin{equation}
    e = E\left( e', f, M_c \right)
\end{equation}
be the eccentricity at merger of a system with chirp mass $M_c$ that has
eccentricity $e'$ at orbital frequency $f$.  Then 
\begin{equation}
    \diff{N}{M_c \dd e' \dd V \dd f} = \diff{N}{M_c \dd e \dd V \dd f} \frac{\partial E}{\partial e'}
\end{equation}
is the binary rate per chirp mass, per eccentricity (at orbital frequency $f$),
per comoving volume, per orbital frequency.  

If the eccentricity evolution satisfies 
\begin{equation}
    \frac{\partial E}{\partial f}\left( e', f, M_c \right) = g\left( E, f \right),
\end{equation}
where $g$ is obtained as the ratio of the right hand sides of Eqs.\
\eqref{eq:dedt}, and \eqref{eq:fdot}, then (differentiating both sides of this
equation by the eccentricity $e'$) the Jacobian 
\begin{equation}
    J \equiv \frac{\partial E}{\partial e'}
\end{equation}
satisfies 
\begin{equation}
    \frac{\partial}{\partial f} J = \frac{\partial g}{\partial E} J.
\end{equation}
This is a linear ODE for $J$, and 

\bibliography{bib}

\end{document}

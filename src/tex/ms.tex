% Define document class
\documentclass[twocolumn]{aastex631}
\usepackage{showyourwork}

% Begin!
\begin{document}

% Title
\title{LISA can measure the rate of eccentric binary black holes (that cannot be) observed by ground-based detectors}

% Author list
\author{@katiebreivik}

% Abstract with filler text
\begin{abstract}
    hype words about lisa, hype words about eccentricity, hype words about bbhs
\end{abstract}

% Main body with filler text
\section{Introduction}
\label{sec:intro}

The environmnetal origins of BBHs are highly degenerate with the eccentricity of the binaries themselves. Thus, a measurement of the rate of eccentric BBHs provides the ability to constrain the relative contributions of isolated and dynamical formation envrionments. 

\section{Rate Derivation}
\label{sec:rates}
Under the assumptions that most BBHs form in orbits with orbital frequencies below $\sim 1~\rm{mHz}$ and that the emission of GWs is the only driver of angular momentum loss in BBHs between the mHz and kHz GW regime, the rate of evolution of BBHs from the mHz to the kHz band can be described analytically using the \citet{Peters1964} evolution equations. Averaging the energy from GW emission, we can write the orbital separation and eccentricity evolution as a function of time as

\begin{equation}
    \Big\langle \frac{\mathrm{d}a}{\mathrm{d}t} \Big\rangle = - \frac{64}{5} \frac{G^3 m_1 m_2 (m_1 + m_2)}{c^5 a^3 (1 - e^2)^{7/2}} \Big(1 + \frac{73}{24} e^2 + \frac{37}{96} e^4 \Big)
    \label{eq:dadt}
\end{equation}

\noindent and 

\begin{equation}
    \Big\langle\frac{\mathrm{d}e}{\mathrm{d}t}\Big\rangle = -\frac{304}{15} e \frac{G^3 m_1 m_2 (m_1 + m_2)}{c^5 a^4 (1-e^2)^{7/2}} \Big( 1 + \frac{121}{304} e^{2}\Big).
    \label{eq:dedt}
\end{equation}

Using Kepler's third law, the orbital separation time evolution can be converted to an orbital frequency evolution 

\begin{equation}
    \Big\langle\frac{\mathrm{d}f}{\mathrm{d}t}\Big\rangle = \frac{48 n}{5 \pi} \frac{\Big(G \mathcal{M}_c \Big)^{5/3}}{c^5} (2 \pi f_{\rm orb})^{11/3} F(e), 
    \label{eq:fdot}
\end{equation}

\noindent where 

\begin{equation}
    F(e) = \frac{1 + \frac{73}{24} e^2 + \frac{37}{96} e^4}{(1 - e^2)^{7/2}}.
    \label{eq:eccentricity_enhancement_factor}
\end{equation}

The comoving merger rate density per unit chirp mass, $\mathcal{M}_c$ of BBHs can then be written as

\begin{equation}
    \frac{\mathrm{d}^2n}{\mathrm{d}\mathcal{M}_c \mathrm{d}t} = \mathcal{R}(\mathcal{M}_c) p(\mathcal{M}_c),
    \label{eq:merger_rate_density}
\end{equation}

\noindent where $p(\mathcal{M}_c)$ is the chirp mass distribution observed by LVK. 

This can be converted into a number of BBHs per unit mass, redshift, and frequency as

\begin{equation}
    \frac{\mathrm{d}^3N}{\mathrm{d}\mathcal{M}_c \mathrm{d}z \mathrm{d}f} = \frac{\mathrm{d}^2n}{\mathrm{d}\mathcal{M}_c \mathrm{d}t} \frac{\mathrm{d}V}{\mathrm{d}z} \frac{\mathrm{d}t}{\mathrm{d}f},
    \label{eq:rate_per_freq}
\end{equation}

where $\mathrm{d}t / \mathrm{d}f$ is just the inverse of equation \ref{eq:fdot}.




\bibliography{bib}

\end{document}
